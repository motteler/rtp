
\documentclass[11pt]{article}

\title{{\bf RTP Format Specification\\
                and\\
        User's Guide}\bigskip \\
        {Version 2.21}\bigskip \\ }

\author{Howard~E.~Motteler and Scott~Hannon}

\date{\today}

\begin{document}

\maketitle

\begin{abstract}

  \noindent
  We present a data format for driving radiative transfer
  calculations and manipulating atmospheric profiles.  Calculated
  and observed radiances may be included as optional fields, for
  more general applications.  An implementation as HDF4 Vdatas is
  given, including Fortran, C, and Matlab application interfaces.

\end{abstract}

% \newpage

%---------------------
\section{Introduction}
%---------------------

The ``Radiative Transfer Profile'' (RTP) format is a data format for
sets of atmospheric profiles, optionally paired with calculated
and/or observed radiances.  The format consists of a header record
and an array of profile records.  It was derived from the GENLN2
user profile format, extened with selected AIRS level~2 field
definitions.  RTP is implemented as HDF4 Vdatas.

The format is intended to give a well-defined interface to radiative
transfer codes, allowing for the specification of just the
information needed for such calculations.  It allow for modularity
of both radiative transfer codes and of other tools for manipulating
profiles, including tools for field selection, level interpolation
and level-to-layer translations, translation of units, and building
composite profiles from multiple sources.  The RTP specification has
some flexibility in the field set actually saved to disk, both to
save space and to provide compatibility across file versions.  The
optional observation fields may be used to build simple co-location
datasets.

%--------------------------------------
\section{The RTP format definition}
%--------------------------------------

The RTP format consists of a header record with information about
all the profiles in a file, and one or more profiles saved as an
array of records.  Field definitions for the header and profile
records are given below.  These names are both the names of the HDF4
Vdata fields and the Fortran, C, and Matlab structure fields, with
the exception of the constituent arrays, as discussed below.  Fields
are matched by name, and depending on the application, only a subset
of the fields described here need be present in an RTP file.

\subsection{Levels and Layers}

The header field {\tt ptype} flags the profile as being a level
profile, a layer profile, or a pseudo-level profile.  For level
profiles, the temperature and gas constituent fields represent point
values at the specified pressure levels.  For layer profiles these
fields represent integrated values in the space between adjacent
{\tt plevs}.  The {\tt palts} field, if used, is altitudes of the
pressure levels for either level or layer profiles.  The {\tt nlevs}
field is the number of pressure levels.  For layer profiles the
number of layers is ${\tt nlevs} - 1$.  Pseudo-level profiles
contain layer gas consistuents and level temperatures.

A convention that lower indices correspond to lower pressures is
suggested but not required.  The header fields {\tt pmax} and {\tt
pmin} are intended to hold the max and min level pressures over all
profiles, or some upper and lower bound on these values.

\subsection{Constituents}

Constituent fields are named with their HITRAN gas ID's, with {\tt
  gas\_1} water, {\tt gas\_2} CO$_2$, and so on.  A list of HITRAN
gas ID's is given in an appendix.  The header field {\tt glist}
gives a list of the IDs for the constituents present in the file.
The default constituent unit is PPMV.

The Fortran and C application interfaces represent constituents as a
2D array {\tt gamnt} whose rows are layers and whose columns are gas
ID index, rather than as a set of separate fields {\tt gas\_<i>} as
they are actually saved in the file.  The {\tt gas\_<i>} fields are
the columns of the 2D {\tt gamnt} array.

There are a wide variety of constituent units in current use; in
consideration of this we have added a {\tt gunit} array to the
header, assigning a unit code for each constituent and allowing at
least the potential for automatic conversions.  These unit codes are
given in {\tt gas\_units\_code.txt}.

Note that only a small subset of possible constituents are typically
recognized and processed by fast models for radiative transfer
calculation, typically water, ozone, and perhaps methane, CO$_2$,
and CO; see the documentation of the relevant radiative transfer
code for more information.

%%%%%%%%%%%%%%%%%%%%%%%%%%%%%%%%%%%%%%%%%%%%%%%%%%%%%%%%%%%%%%%%%%%%%%%%%%
\begin{figure}
\vskip-14mm
{\footnotesize
\begin{verbatim}
                      RTP Header Fields

 field name   short description       data type         units
 ----------   ------------------     -----------       -------
 ptype        profile type           scalar int32     see note [1]
 pfields      profile field set      scalar int32     see note [2]

 pmin         min plevs value        scalar float32   millibars
 pmax         max plevs value        scalar float32   millibars
 ngas         number of gases        scalar int32     [0,MAXGAS]
 glist        constituent gas list   ngas   int32     HITRAN gas ID
 gunit [3]    constituent gas units  ngas   int32     gas unit code

 pltfid       platform ID            scaler int32     platform code
 instid       instrument ID          scaler int32     instrument code
 nchan        number of channels     scalar int32     count
 ichan        channel numbers        nchan  int32     [0,MAXCHAN]
 vchan        channel center freq.   nchan  float32   cm^-1
 vcmin        channel set min freq.  scalar float32   cm^-1
 vcmax        channel set max freq.  scalar float32   cm^-1

 iudef        user-defined array     MAXIUDEF int32   undefined
 itype        user-defined integer   scaler int32     undefined

Notes:

 [1] ptype values are
        1. level profile       LEVPRO = 0
        2. layer profile       LAYPRO = 1
        3. AIRS pseudo-layers  AIRSLAY = 2

 [2] RTP profile fields are organized in five groups
        1. profile data              PROFBIT = 1
        2. calculated IR radiances   IRCALCBIT = 2
        3. observed IR radiances     IROBSVBIT = 4
     For example, a profile with both calculated and observed IR
     radiances would have pfields = PROFBIT + IRCALCBIT + IROBSVBIT

 [3] For suggested gas units code, see the file gas_units_code.txt

\end{verbatim}
}
\end{figure}
%%%%%%%%%%%%%%%%%%%%%%%%%%%%%%%%%%%%%%%%%%%%%%%%%%%%%%%%%%%%%%%%%%%%%%%%%%
\begin{figure}
{\footnotesize
\begin{verbatim}
                Profile Fields -- Surface Data
  
 field name   short description       data type         units
 ----------   ------------------     -----------       -------
 plat         profile latitude       scalar float32   [-90 to 90] deg.
 plon         profile longitude      scalar float32   [-180 to 360] deg.
 ptime        profile time           scalar float64   TAI  

 stemp        surface temperature    scalar float32   Kelvins
 salti        surface altitude       scalar float32   meters
 spres        surface pressure       scalar float32   millibars
 landfrac     land fraction          scalar float32   [0 to 1]
 landtype     land type code         scalar int32     land code

 wspeed       wind speed             scalar float32   meters/sec

 nemis [1]    number of emis. pts    scalar int32     [0,MAXEMIS]
 efreq [1]    emissivity freq's      nemis  float32   cm^-1
 emis         surface emissivity     nemis  float32   [0 to 1]
 rho          surface reflectance    nemis  float32   [0 to 1]

Notes:

 [1] The nemis and efreq data is also used with cloud emis and rho.



                Profile Fields -- Atmospheric Data

 field name   short description         data type         units
 ----------   ------------------       -----------       -------
 nlevs        number of press lev's    scalar int32     [0,MAXLEV]
 plevs        pressure levels          nlevs  float32   millibars
 palts        level altitudes          nlevs  float32   meters
 ptemp        temperature profile      nlevs  float32   Kelvins
 gas_<i> [1]  gas amount               nlevs  float32   HEAD.gunit
 gtotal       total column gas amount  ngas   float32   undefined
 gxover       gas crossover press      ngas   float32   millibars
 txover       temp crossover press     scalar float32   millibars
 co2ppm       CO2 mixing ratio         scalar float32   PPMV

Notes:

 [1] There is one field here for each constituent in a file; the 
     constituents are listed in the header field glist.  The Fortran 
     and C APIs presents this data as [ngas x nlevs] array gamnt.

\end{verbatim}
}
\end{figure}
%%%%%%%%%%%%%%%%%%%%%%%%%%%%%%%%%%%%%%%%%%%%%%%%%%%%%%%%%%%%%%%%%%%%%%%%%%
\begin{figure}
{\footnotesize
\begin{verbatim}
              Profile Fields -- Cloud Data
  
 field name   short description       data type         units
 ----------   ------------------     -----------       -------
 clrflag      clear flag             scalar int32     [0,1] or clear code

 tcc          total cloud cover      scalar float32   [0 to 1]
 cc           cloud cover            nlevs float32    [0 to 1]
 ciwc         cloud ice water        nlevs float32    g/g 
 clwc         cloud liquid water     nlevs float32    g/g 

 ctype   [1]  cloud type code        scalar int32     cloud code
 cfrac   [2]  cloud fraction         scalar float32   [0 to 1]
 cemis   [2]  cloud top emissivity   nemis  float32   [0 to 1]
 crho    [2]  cloud top reflectance  nemis  float32   [0 to 1]
 cprtop  [2]  cloud top pressure     scalar float32   millibars
 cprbot       cloud bottom pressure  scalar float32   millibars
 cngwat       cloud non-gas water    scalar float32   g/m^2
 cpsize       cloud particle size    scalar float32   microns
 cstemp  [2]  cloud surface temp     scalar float32   Kelvins

 ctype2  [1]  cloud2 type code       scalar int32     cloud code
 cfrac2  [2]  cloud2 fraction        scalar float32   [0 to 1]
 cemis2  [2]  cloud2 top emissivity  nemis  float32   [0 to 1]
 crho2   [2]  cloud2 top reflectance nemis  float32   [0 to 1]
 cprtop2 [2]  cloud2 top pressure    scalar float32   millibars
 cprbot2      cloud2 bottom pressure scalar float32   millibars
 cngwat2      cloud2 non-gas water   scalar float32   g/m^2
 cpsize2      cloud2 particle size   scalar float32   microns
 cstemp2 [2]  cloud2 surface temp    scalar float32   Kelvins
 cfrac12      cloud1+2 fraction      scalar float32   [0 to 1]

Notes:

 [1] For suggested cloud type codes see file cloud_code.txt
 [2] These cloud fields may instead be used for alternate surfaces.
    
\end{verbatim}
}
\end{figure}
%%%%%%%%%%%%%%%%%%%%%%%%%%%%%%%%%%%%%%%%%%%%%%%%%%%%%%%%%%%%%%%%%%%%%%%%%%
\begin{figure}
{\footnotesize
\begin{verbatim}
              Profile Fields -- Orientation Data
  
 field name   short description       data type         units
 ----------   ------------------     -----------       -------
 pobs         observer pressure      scalar float32   millibars
 zobs         observer height        scalar float32   meters
 upwell       radiation direction    scalar int32     1=up, 2=down
 scanang      IR scan/view angle     scalar float32   [-90 to 90] deg.
 satzen       IR zenith angle        scalar float32   [0 to 180] deg.
 satazi       IR azimuth angle       scalar float32   [-180 to 180] deg.

 solzen       sun zenith angle       scalar float32   [0 to 180] deg.
 solazi       sun azimuth angle      scalar float32   [-180 to 180] deg.
 sundist      sun-Earth distance     scalar float32   meters
 glint        glint distance         scalar float32   meters

\end{verbatim}
}
\end{figure}
%%%%%%%%%%%%%%%%%%%%%%%%%%%%%%%%%%%%%%%%%%%%%%%%%%%%%%%%%%%%%%%%%%%%%%%%%%
\begin{figure}
{\footnotesize
\begin{verbatim}
              Profile Fields -- Radiance Data
  
 field name   short description       data type         units
 ----------   ------------------     -----------       -------
 rlat         radiance obs lat.      scalar float32   [-90 to 90] deg.
 rlon         radiance obs lon.      scalar float32   [-180 to 360] deg. 
 rtime        radiance obs time      scalar float64   TAI

 findex       file (granule) index   scalar int32     index
 atrack       along-track index      scalar int32     index
 xtrack       cross-track index      scalar int32     index
 ifov         field of view index    scalar int32     index

 robs1        observed IR rad.       nchan  float32   mW/m^2/cm^-1/str
 calflag      calibration flag       nchan  uint8     see text
 robsqual     radiance quality       scalar int32     undefined
 freqcal      frequency calibration  scalar float32   undefined

 rcalc        calculated IR rad.     nchan  float32   mW/m^2/cm^-1/str

\end{verbatim}
}
\end{figure}
%%%%%%%%%%%%%%%%%%%%%%%%%%%%%%%%%%%%%%%%%%%%%%%%%%%%%%%%%%%%%%%%%%%%%%%%%%
\begin{figure}
{\footnotesize
\begin{verbatim}
              Profile Fields -- User Defined Data
  
 field name   short description       data type         units
 ----------   ------------------     -----------       -------
 pnote        profile annotation     MAXPNOTE uint8   text or undefined
 udef         user-defined array     MAXUDEF float32  undefined
 iudef        user-defined array     MAXIUDEF int32   undefined
 itype        user-defined integer   scalar int32     undefined

\end{verbatim}
}
\end{figure}
%%%%%%%%%%%%%%%%%%%%%%%%%%%%%%%%%%%%%%%%%%%%%%%%%%%%%%%%%%%%%%%%%%%%%%%%%%

\subsection{Field Sets and Sizes}

The RTP interface uses the header size fields {\tt mlevs}, {\tt
  memis}, {\tt ngas}, {\tt nchan}, and {\tt pfields} to build new,
generally much more compact, field lists for the header and profile
Vdata buffers.  Whenever a header size field is zero, the relevant
header or profile field is dropped from the Vdata.  The header size
fields {\tt mlevs}, {\tt memis}, {\tt ngas}, and {\tt nchan} all
default to zero, and should be less than or equal to the
corresponding max limits {\tt MAXLEV}, {\tt MAXEMIS}, {\tt MAXGAS},
and {\tt MAXCHAN}.

Individual profiles have pressure levels set with {\tt nevs}, and
emissivity and reflectance sets with {\tt nemis}, both of which
default to zero.  These should be less than or equal to the header
limits {\tt mlevs} and {\tt memis}.  For most applications they can
just be set to {\tt mlevs} and {\tt memis}.  All profiles in a file
are assumed to have the same constituent set, and if radiances are
present all profiles have the same channel set.  Most arrays have an
associated size field.  If this size field is in the header, as in
the case of {\tt ngas} or {\tt nchan} then it is assumed to be the
same for all profiles, while if the size field is in a profile, as
in the case of {\tt nlevs} or {\tt nemis}, then it applies only to
that profile.

% \pagebreak
\subsection{Field Groups}

The {\tt pfields} field in the header is used by the C/Fortran API
to control what which field groups will be written to a file.  Profile
fields are organized as three groups,

{\small
\begin{verbatim}
   1. profile data                     PROFBIT = 1
   2. calculated radiances           IRCALCBIT = 2
   3. observed radiances             IROBSVBIT = 4
\end{verbatim}
}

These groups can occur in any combination.  The associated numbers
are bit fields, to be set in pfields if the associated data is
present in the file.  Thus for example profile data with calculated
and observed radiances would be represented as {\tt pfields}~$=$
{\tt PROFBIT}~$+$ {\tt IRCALCBIT}~$+$ {\tt IROBSVBIT}.

Note that we can have {\tt nchan}~$>0$ and channel data in the
header without having either calculated or observed radiances in a
file, to specify a set of channels whose radiances are to be
calculated later.

\subsection{HDF Attributes}

Attributes are associated either with the header or with the
profile record set, and have three parts: the field the attribute
is associated with, the attribute name, and the attribute text.  In
addition to proper field names, the field name ``header'' is used
for general header attributes, and ``profile'' for general profile
attributes.

RTP attributes should typically include such information as title,
author, date, and at least a brief descriptive comment.  This
general information should be set as attributes of the header
record.  Note that the Fortran/C API uses the 2D {\tt gamnt} array
for constituents; this is not actually a Vdata field, and so can not
take an attribute.  Attributes may be attached to individual
constituents with their {\tt gas\_<i>} names, where {\tt <i>} is the
HITRAN gas ID.

\section{Application Interfaces}

\subsection{The Fortran and C API}

The Fortran API consists of four routines: {\tt rtpopen}, {\tt
rtpread}, {\tt rtpwrite}, and {\tt rtpclose}.  Documentation for
these is included in an appendix.  The Fortran API uses static
structures whose fields, with a few exceptions noted below, are the
same as the RTP fields defined above.  Normally, only a subset of
the Fortran structure fields will be written, with the header field
{\tt pfields} and the header size fields used to determine what
actually goes into a file.  When reading data, if a file contains
header or profile fields not in the Fortran structure definition,
they are simply ignored.  Fields that are defined in the Fortran
structure but are not in a file are returned as ``BAD'', or with the
first element BAD, for vectors, while missing size fields are
returned as zero.

Attributes are passed to and from the Fortran API in the {\tt
RTPATTR} structure array.  The records in this array have three
fields: {\tt fname}, the field name the attribute is to be
associated with, {\tt aname}, the attribute name, and {\tt atext},
the attribute text.  The header attribute field name should be
either ``header'', for a general attribute or comment, or a
particular header field name.  Similarly, the attribute profile
field name should be either ``profiles'' or a specific profile
field.  Attribute strings need to be null-terminated, with char(0),
and the record after the last valid record in an attribute set
should have fname set to char(0).  See ftest1.f for and ftest2.f
examples of reading, writing, and updating attributes.

The Fortran structures differer from the Vdata fields in two ways.
First, instead of a {\tt gas\_<i>} profile field for each
constituent, the Fortran API uses a single array {\tt
gamnt(MAXLEV,MAXGAS)} to pass constituent amounts; the {\tt
gas\_<i>} fields from the HDF file are the columns of this array.
Second, the Fortran/C RTP header structure includes the following
max size fields, which are not actually written to the Vdata header.

{\small
\begin{verbatim}
 mlevs     max number of levels   scalar int32   [0,MAXLEV]
 memis     max num of emis pts    scalar int32   [0,MAXEMIS]
\end{verbatim}
}

On a read, these fields are set to the associated profile Vdata
field sizes.  On a write, they are used to to set the size of the
associated Vdata profile fields.  They can simply be set to the MAX
limits, or to zero if the fields are not used; but using an actual
max for the profile set, particularly for mlevs, can give a
significant space savings.

A Makefile is supplied to build the RTP API routines as a library
file librtp.a, along with some C and Fortran test programs and
utilities.

\subsection{The Matlab API}

The RTP Matlab implementation is a fairly direct mapping between
Matlab structure arrays and HDF~4 Vdatas.  A read will only return
those fields that are in the HDF Vdata, and a write will only write
the fields in the Matlab structure.  The Matlab RTP API is available
as part of the ASL package h4tools; see the README file there for
more information.  The key routines are rtpread.m and rtpwrite.m.
These Matlab functions are reasonably efficient, but there is a key
difference between the Matlab and Fortran/C interfaces---the
Fortran/C interface loops on records, with only minimal buffering
needed, while the Matlab interface loads the entire file in memory,
for both reads and writes.

% \subsection{Data Types}
% 
% Most RPT fields are either 32-bit integers or 32-bit floats, as
% noted in the field tables, with the exception of the time fields
% which are 64-bit floats, and the pnote and calflag fields, which are
% uint8 character arrays (as of 21 October 2011; they were previously
% char).  The HDF C types are defined in the HDF include file ``hdf.h''.
% 
% {\footnotesize
% \begin{verbatim}
%    HDF type codes     HDF C types      Fortran types
%     DFNT_INT32          int32           integer*4
%     DFNT_FLOAT32        float32         real*4
%     DFNT_FLOAT64        float64         real*8
%     DFNT_CHAR8          char8           character*<n>
%     DFNT_UCHAR8         uchar8          character*<n>
%     DFNT_UINT8          uchar8          character*<n>
% \end{verbatim}
% }

\newpage
\section{Fortran API}
{\footnotesize
\begin{verbatim}

NAME

  rtpopen -- Fortran interface to open RTP files

SUMMARY

  rtpopen() is used to open an HDF RTP ("Radiative Transfer Profile")
  file for reading or writing profile data.  In addition, it reads or
  writes RTP header data and HDF header and profile attributes. 

FORTRAN PARAMETERS

  data type             name      short description       direction
  ---------             -----     -----------------       ---------
  CHARACTER *(*)        fname     RTP file name              IN
  CHARACTER *(*)        mode      'c'=create, 'r'=read       IN
  STRUCTURE /RTPHEAD/   head      RTP header structure       IN/OUT
  STRUCTURE /RTPATTR/   hfatt     RTP header attributes      IN/OUT
  STRUCTURE /RTPATTR/   pfatt     RTP profile attributes     IN/OUT
  INTEGER               rchan     RTP profile channel        OUT

VALUE RETURNED

  0 if successful, -1 on errors

INCLUDE FILES

  rtpdefs.f -- Fortran header, profile, and attribute structures

DISCUSSION

  The valid open modes are 'r' to read an existing file and 'c' to
  create a new file.

  HDF attributes are read and written in an array of RTPATTR
  structures, with one structure record per attribute.  Attributes
  should be terminated with char(0), and are returned that way, for
  a read.  The end of the attribute array is flagged with a char(0)
  at the beginning of the fname field.

\end{verbatim}
}

\newpage
{\footnotesize
\begin{verbatim}
NAME    

  rtpread -- Fortran interface to read an RTP profile

SUMMARY

  rtpread reads a profile from an open RTP channel, and returns
  the data in the RTPPROF structure.  Successive calls to rtpread
  return successive profiles from the file, with -1 returned on
  EOF.


FORTRAN PARAMETERS

  data type             name      short description       direction
  ---------             -----     -----------------       ---------
  INTEGER               rchan     RTP profile channel         IN
  STRUCTURE /RTPPROF/   prof      RTP profile structure       OUT

VALUE RETURNED

  1 (the number of profiles read) on success , -1 on errors or EOF

\end{verbatim}
}

{\footnotesize
\begin{verbatim}

NAME    

  rtpwrite -- Fortran interface to write an RTP profile

SUMMARY

  rtpwrite writes an RTP profile, represented as the contents
  of an RTPPROF structure, to an open RTP channel.  Successive
  calls write successive profiles.

FORTRAN PARAMETERS

  data type             name      short description       direction
  ---------             -----     -----------------       ---------
  INTEGER               rchan     RTP profile channel         IN
  STRUCTURE /RTPPROF/   prof      RTP profile structure       IN

VALUE RETURNED

  0 on success, -1 on errors

\end{verbatim}
}

\newpage
{\footnotesize
\begin{verbatim}
NAME    

  rtpclose -- Fortran interface to close an RTP open channel

SUMMARY

  rtpclose finishes up after reading or writing an RTP file, 
  writing out any buffers and closing the HDF interface

FORTRAN PARAMETERS

  data type     name      short description      direction
  ---------     -----     -----------------      ---------
  INTEGER       rchan     RTP profile channel       IN

VALUE RETURNED

  0 on success, -1 on errors



NAME    

  rtpinit -- initialze RTP profile and header structures

SUMMARY

  rtpinit initializes RTP profile structures with some sensible
  default vaules, and is used when creating a new profile set; it
  should generally not be used when modifying existing profiles.

  rtpinit sets all field sizes to zero, and all data values to
  "BAD", so that only actual values and sizes need to be written

FORTRAN PARAMETERS

  data type             name      short description       direction
  ---------             -----     -----------------       ---------
  STRUCTURE /RTPHEAD/   head      RTP header structure       OUT
  STRUCTURE /RTPPROF/   prof      RTP profile structure      OUT

VALUE RETURNED

  rtpinit always returns 0

\end{verbatim}
}

\section{Modification}

The RTP parameters and field sets can be modified, but some caution
is in order.  Parameters and structures are defined in rtp.h, in the
include directory.  The corresponding Fortran declarations are in
rtpdefs.f.  Declarations that reserve space (initialized structures)
are in rtpdef.h.  These must all be in agreement.  So for example we
would set the parameter MAXLEV to 120 in both rtp.h and rtpdefs.f.

The C/Fortran API uses C and Fortran structs as an interface to the
RTP files.  This is convenient, but the structs are static, with a
fixed set of fields, while the format of an RTP file can vary; for
example with different sets of fields, or different field sizes.  To
make this work the interface uses both a static, conventional record
structure and a list of field names as strings, with field size and
data type.  We call the latter data structure FLISTs.  The string
names are used to match fields in the file with positions in the
static buffer structure.  The C interface structures are defined in
rtp.h, the corresponding Fortran structures in rtpdefs.f, and the
associated FLISTs in rtpdef.h.  For example, the field {\tt glist}
might appear in rtp.h, in the structure rtp\_head, as
\begin{verbatim}
    int32       glist[MAXGAS];   /* constituent gas list */
\end{verbatim}
\noindent 
The corresponding line in rtpdefs.f, in the Fortran structure
/RTPHEAD/, would be
\begin{verbatim}
    integer*4   glist(MAXGAS)    ! constituent gas list
\end{verbatim}
\noindent
{\tt glist} also needs to be set in rtpdef.h, in the struct FLIST
hfield, with the line
\begin{verbatim}
   "glist",     DFNT_INT32,     MAXGAS,
\end{verbatim}
\noindent
The positions of the fields should be the same in the C and Fortran
structs, the FLISTs, and in the documentation.  The parameters
NHFIELD and NPFIELD (the number of header and profile fields) should
be updated to reflect added or deleted fields.  Finally, fields
whose size is set with the header values {\tt memis}, {\tt mlevs},
{\tt ngas}, {\tt nchan}, and {\tt pfields} require modification of
rtpwrite1.c; look for similar cases there as a pattern.

\newpage
\section{Gas IDs}
{\footnotesize
\begin{verbatim}


                   HITRAN Gas ID List
                   ------------------
                                    
    Gases from the 2008 HITRAN line database
    
     1 = H2O (water vapor)          25 = C2H2 (acetylene)
     2 = CO2                        26 = C2H2 (ethane)
     3 = O3 (ozone)                 27 = PH3
     4 = N2O                        28 = PH3
     5 = CO                         29 = COF2
     6 = CH4 (methane)              30 = SF6
     7 = O2 (oxygen)                31 = H2S
     8 = NO                         32 = HCOOH
     9 = SO2                        33 = HO2
    10 = NO2                        34 = O
    11 = NH3 (ammonia)              35 = ClONO2 (also see 61)
    12 = HNO3 (nitric acid)         36 = NO+
    13 = OH                         37 = HOBr
    14 = HF                         38 = C2H4
    15 = HCl                        39 = CH3OH
    16 = HBr                        40 = CH3Br
    17 = HI                         41 = CH3CN
    18 = ClO                        42 = CF4 (also see 54)
    19 = OCS
    20 = H2CO
    21 = HOCl
    22 = N2 (nitrogen)
    23 = HCN
    24 = CH3Cl
    25 = H2O2

\end{verbatim}
}
\newpage
{\footnotesize
\begin{verbatim}

             Non-standard Gas ID Lists
             -------------------------

    Gases represented by cross-sections
 
    51 = CCl3F (CFC-11)             66 = CHClFCF3 (HCFC-124)
    52 = CCl2F2 (CFC-12)            67 = CH3CCl2F (HCFC-141b)
    53 = CClF3 (CFC-13)             68 = CH3CClF2 (HCFC-142b)
    54 = CF4 (CFC-14)               69 = CHCl2CF2CF3 (HCFC-225ca)
    55 = CHCl2F (CFC-21)            70 = CClF2CF2CHClF (HCFC-225cb)
    56 = CHClF2 (CFC-22)            71 = CH2F2 (HFC-32)
    57 = C2Cl3F3 (CFC-113)          72 = CHF2CF3 (HFC-134a)
    58 = C2Cl2F4 (CFC-114)          73 = CF3CH3 (HFC-143a)
    59 = C2ClF5 (CFC-115)           74 = CH3CHF2 (HFC-152a)
    60 = CCl4
    61 = ClONO2 (also see 35)
    62 = N2O5
    63 = HNO4
    64 = C2F6
    65 = CHCl2CF3 (HCFC-123)

    Special purpose IDs
    101 self-broadened H2O continuum
    102 foreign-broadened H2O continuum
    201 Cloud one
    202 Cloud two
    203 Cloud three

\end{verbatim}
}

\end{document}

